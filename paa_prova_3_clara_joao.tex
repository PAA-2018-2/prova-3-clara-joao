\documentclass[12pt]{article}

\usepackage[utf8]{inputenc}
\usepackage[brazil]{babel}
\usepackage{amsmath,amssymb}
\usepackage{pdfsync}
\usepackage[all]{xy}
\usepackage{color}

\newcommand{\resposta}[1]{ \noindent {\bf Solução}. {\color{blue} #1}}

\title{{\large Universidade de Brasília \\ Instituto de Ciências Exatas \\
Departamento de Ciência da Computação} \\[1cm]
CIC 117536 - Projeto e Análise de Algoritmos \\[.5cm]  Terceira Prova \\[.5cm] Turma: B}
\author{{\bf NP-completude}}
\date{Prof. Flávio L. C. de Moura \\[.5cm] \today}

\begin{document}
\maketitle

\begin{enumerate}
\item {\bf (2.5 pontos)} O problema 2-SAT tem como instâncias as
  fórmulas lógicas formadas por conjunções de disjunções de até dois
  literais, onde um literal é uma variável booleana ou a negação de
  uma variável booleana. Por exemplo, a expressão a seguir é uma
  instância de 2-SAT:

  $$(x_1\lor \neg x_2)\land (\neg x_1 \lor \neg x_3) \land (x_1 \lor x_2) \land x_3$$

  Prove que 2-SAT $\in$ P.

 
  \resposta{Dizer que uma instância de 2-SAT $\in$ P significa encontrar um algoritmo que descubra se existem ou não valores para os literais tais que a fórmula final seja verdadeira, e fazê-lo em tempo polinomial.
  
  Por ser composta de conjunções, a fórmula é dita verdadeira se cada uma de suas cláusulas é verdadeira. Isso significa que $(x_1\lor \neg x_2)$, $(\neg x_1 \lor \neg x_3)$, $(x_1 \lor x_2)$ e $x_3$, neste caso, devem ser todas verdadeiras.
  
  Para que uma fórmula arbitrária $(A \lor B)$ seja verdadeira, temos que:
  
  \begin{enumerate}
      \item Se A = 0, B deve ser 1
      \item Se B = 0, A deve ser 1
  \end{enumerate}
  
  Sendo assim, substituímos $(A \lor B)$ por $(\neg A \implies B) \land (\neg B \implies A) $ e criamos um grafo de inferências, onde cada literal se torna dois vértices (ele e sua negação) e cada aresta equivale a uma implicação das definidas acima para toda a conjunção.
  
  Com esse grafo, temos diversas novas inferências, pois se $A \implies B$ e $B \implies C$, então $A \implies C$. Isso significa que cada caminho novo entre vértices se torna uma implicação.
  
  Assim, basta avaliarmos as componentes fortemente conexas do grafo construído e verificar se uma variável e sua negação estão ou não presentes em alguma delas. A presença significa dizer que $A \implies \neg A$ e $\neg A \implies A$, um absurdo: não há valores para os literais que tornam a instância verdadeira; e a ausência significa o contrário, que há valores que o façam.
  
  Sendo assim, 2-SAT pode ser feito em tempo polinomial, pois como já visto anteriormente, calcular componentes fortemente conexas pode ser feito em O(V+E), e depois de realizar este passo, basta verificar se um literal e sua negação estão ao mesmo tempo em alguma das SCC com algum algoritmo de busca, também polinomial.

  }
  
\item {\bf (2.5 pontos)} Em aula, assumimos que SAT é um problema
  NP-completo (Teorema de Cook-Levin), e a partir deste fato mostramos
  que 3-SAT e CLIQUE também são problemas NP-completos. As reduções
  foram feitas de acordo com o seguinte diagrama:

  $$\xymatrix{
    SAT \ar[d] \\
    3\mbox{-}SAT \ar[d] \\
    CLIQUE 
  }$$
  
  Um ciclo Hamiltoniano é um ciclo simples que visita cada vértice de
  um grafo exatamente uma vez. Considere o problema de decisão
  HAM-CYCLE que pergunta se um dado grafo (não-dirigido) $G$ possui um
  ciclo Hamiltoniano. Mostre que HAM-CYCLE é um problema
  NP-completo. Sua solução deve ser construída a partir de SAT, 3-SAT
  ou CLIQUE. Caso, você não veja como reduzir diretamente HAM-CYCLE a
  partir destes, mas sabe como fazê-lo a partir de um certo problema
  $Q$ então inicialmente mostre que $Q$ é NP-completo a partir de SAT,
  3-SAT ou CLIQUE, e assim por diante. Digamos que você não saiba como
  mostrar que $Q$ é NP-completo diretamente a partir de SAT, 3-SAT ou
  CLIQUE, mas você sabe como fazê-lo a partir de outro problema $Q'$,
  e também sabe como mostrar que $Q'$ é NP-completo a partir de 3-SAT,
  por exemplo. Então o diagrama correspondente à sua solução seria:

$$\xymatrix{
  SAT \ar[d] & & & \\
  3\mbox{-}SAT \ar[d]\ar[r] & Q' \ar[r] & Q \ar[r] & \mbox{HAM-CYCLE}  \\
  CLIQUE & & & 
}$$

  E todas as reduções (de 3-SAT para $Q'$, de $Q'$ para $Q$ e de $Q$ para HAM-CYCLE) devem ser detalhadas na sua solução.

\resposta{

  Para provar que HAM-CYCLE é um problema NP-completo, precisamos satisfazer duas condições: provar que está em NP e que é NP-difícil.

  A primeira parte do problema é facilmente solucionada, tendo em vista que seria possível criar um algoritmo em tempo polinomial que descobre se um grafo contém um ciclo hamiltoniano: basta atribuirmos cores aos vértices (para ter certeza de que só visitamos cada um deles uma única vez) e percorrer todos os possíveis caminhos entre um vértice e ele mesmo, fazendo as devidas alterações para a limitação da visitação única, verificando se em alguma dessas respostas todos os vértices foram visitados (como um caixeiro-viajante para todos os nós).
  
  Para a segunda parte do problema, precisamos converter qualquer ins-\\tância de um problema conhecido no espaço de NP-completos em um HAM-CYCLE em tempo polinomial. Vamos provar transformando um 3-SAT em um ciclo hamiltoniano.
  
  Criamos, então, um grafo a partir de uma instância 3-SAT, onde cada literal se torna um vértice e uma cláusula é uma ligação entre vértices. Com isso, montamos as arestas do grafo de forma que a instância 3-SAT é verdadeira se conseguimos percorrer o grafo em uma direção só chegando ao final no vértice inicial. Isso configura um ciclo hamiltoniano e prova que 3-SAT pode ser reduzido a HAM-CYCLE. Como 3-SAT é NP-completo e 3-SAT $\leq$ \textsubscript{p} HAM-CYCLE, então HAM-CYCLE é NP-difícil, segundo teorema dado em sala.
  
  }
  
\item {\bf (2.5 pontos)} Considere o seguinte jogo em um grafo
  (não-dirigido) $G$, que inicialmente contém 0 ou mais bolas de gude
  em seus vértices: um movimento deste jogo consiste em remover duas
  bolas de gude de um vértice $v\in G$, e adicionar uma bola a algum
  vértice adjacente de $v$. Agora, considere o seguinte problema: Dado
  um grafo $G$, e uma função $p(v)$ que retorna o número de bolas de
  gude no vértice $v$, existe uma sequência de movimentos que remove
  todas as bolas de $G$, exceto uma? Mostre que este problema é
  NP-completo. A mesma observação feita no exercício anterior vale
  aqui: a prova deve ser feita a partir de problemas que provamos
  serem NP-completos, e reduções intermediárias, caso existam, devem
  ser incluídas na solução.

  \resposta{
    Para provar que este problema é NP-completo, precisamos provar que pertence a NP e é NP-difícil.
    
    Inicialmente, para provar que pertence a NP, precisamos criar um algoritmo que adivinhe uma solução para o problema e, em seguida, verifique se esta é correta.
    
    Assim, a partir de uma sequência de jogadas determinada, testamos uma a uma e realizamos a verificação final: num grafo com cada vértice contendo o atributo ``número de bolas'', retiramos 2 do vértice escolhido inicialmente e adicionamos 1 a um vértice adjacente. Repetimos este passo até o fim da sequência, verificando ao final se restou apenas 1 bola no total dos vértices. Se não houver, refazemos a sequência para outros vértices adjacentes possíveis a serem adicionados. Ao final de testar todas as possibilidades, se a verificação ainda não for satisfatória, aquela sequência não é possível para o jogo, e em qualquer momento que ela for satisfatória, o algoritmo termina porque a sequência é correta.
    
    Por fim, para mostrar que o problema das bolas de gude é NP-difícil, precisamos mostrar que algum NP-completo pode ser reduzido a ele. Isso pode ser mostrado com HAM-CYCLE, já que provamos que este é NP-completo na questão anterior. Para isso, a partir de um HAM-CYCLE, queremos criar um problema de bolas de gude. Temos que se o algoritmo de HAM-CYCLE é rodado em um grafo e a saída é que existe um ciclo hamiltoniano nele, precisamos que exista uma sequência de jogadas que resulte em uma única bola de gude na mesa ao final dela, e vice-versa.
    
    Um ciclo hamiltoniano pode ser descrito como o problema das bolas de gude suficientemente se cada vértice deste último for visitado. Isso ocorre pois sempre formamos um ciclo no problema das bolas de gude: cada jogada consiste basicamente em visitar um nó e ir a seu adjacente. Ao fim da sequência, se todos os vértices contiverem 0 bolas de gude e 1 deles contiver uma, significa que fechamos um ciclo e visitamos cada vértice uma única vez, ou seja, temos um ciclo hamiltoniano.
    
  }
  
\item {\bf (2.5 pontos)} Uma fórmula booleana em {\it forma normal conjuntiva com disjunção exclusiva (FNCX)} é uma conjunção de diversas cláusulas, e cada cláusula é uma disjunção exclusiva (XOR) de diversos literais. Lembre-se que a disjunção exclusiva é dada por:

  $$\begin{array}{|l|l|l|}\hline
      a & b & a \oplus b \\ \hline
      V & V & F \\ \hline
      V & F & V \\ \hline
      F & V & V \\ \hline
      F & F & F \\ \hline
  \end{array}$$

  O problema FNCX-SAT pergunta se uma dada fórmula em FNCX é
  satisfatível. Mostre que o problema FNCX-SAT está em $P$, ou então
  que FNCX-SAT é NP-completo. No último caso, a mesma observação feita
  nos dois exercícios anteriores vale aqui: a prova deve ser feita a
  partir de problemas que provamos serem NP-completos, e reduções
  intermediárias, caso existam, devem ser incluídas na solução.

  \resposta{
    A partir da tabela-verdade da disjunção exclusiva, percebe-se que esta pode ser lida como uma disjunção de conjunções, como a seguir:
    
    $$ A \oplus B = A \overline{B} + \overline{A} B $$
    
    Essa transformação pode ser feita em tempo linear, visto que apenas realiza atribuições a cada cláusula, transformando-as em disjunções de variantes dos literais (negados ou não).
    
    Assim, pensamos em $A\overline{B} = A\textsubscript{1}$ e $\overline{A}B = A\textsubscript{2}$ e a cláusula pode ser vista como ($A\textsubscript{1} \lor A\textsubscript{2}$).
    
    Então, temos que um problema FNCX-SAT reduz para um problema SAT em tempo polinomial, como visto acima. Sendo assim, FNCX-SAT $\in$ P: basta reduzirmos as disjunções exclusivas em disjunções de conjunções, e então rodar o mesmo algoritmo utilizado para problemas SAT.
    
  }
\end{enumerate}

\end{document}

%%% Local Variables:
%%% mode: latex
%%% TeX-master: t
%%% End: